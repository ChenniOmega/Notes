\documentclass[12pt]{report}
\newtheorem{defn}{Definition}[section]
\newtheorem{thm}{Theorem}[section]
\newtheorem{lemma}{Lemma}[section]
\newenvironment{mydefn}{\begin{defn} \begin{upshape}}{\end{upshape} \end{defn}}
\newenvironment{proof}
       {\begin{flushleft} \begin{description}
              \item \textit{\textbf{Proof:}}}
        {\hfill\rule{2.1mm}{2.1mm}
              \end{description}\end{flushleft}}
%\newenvironment{myexample}{\begin{flushleft}
%i would like to make defn in a slightly different font to distinguish it from normal text
\usepackage{amsmath, amssymb, mathrsfs, graphicx}
\usepackage{algpseudocode}
\usepackage[T1]{fontenc}
%\usepackage{natbib}
\algnotext{EndFor}
\addtolength{\textwidth}{100pt}
\addtolength{\oddsidemargin}{-50pt}
\addtolength{\evensidemargin}{-50pt}
\linespread{1.2}

\def\pd{{\partial}}
\def\d{{\mbox{d}}}

\def\grad {{\nabla}}
\def\al{{\alpha}}
\def\be{{\beta}}
\def\ga{{\gamma}}
\def\muv{{\overrightarrow{\mu}}}

\def\bfe{{\mathbf{e}}}
\newcommand{\Fbf}{\mathbf{F}}
\newcommand{\fbf}{\mathbf{f}}
\newcommand{\nbf}{\mathbf{n}}
\newcommand{\Pbf}{\mathbf{P}}
\def\bfp{{\mathbf{p}}}
\newcommand{\pbf}{\mathbf{p}}
\newcommand{\Ubf}{\mathbf{U}}
\def\bfu{{\mathbf{u}}}
\newcommand{\Xbf}{\mathbf{X}}
\newcommand{\xbf}{\mathbf{x}}
\def\bfx{{\mathbf{x}}}
\newcommand{\Ybf}{\mathbf{Y}}
\newcommand{\ybf}{\mathbf{y}}
\def\bfy{{\mathbf{y}}}

\def\bfmu{{\mathbf{\mu}}}

\def\Lie{{\mathcal{L}}}
\newcommand{\Pcal}{\mathcal{P}}
\newcommand{\Mcal}{\mathcal{M}}
\newcommand{\Dcal}{\mathcal{D}}
\newcommand{\Ncal}{\mathcal{N}}
\newcommand{\Ucal}{\mathcal{U}}

\newcommand{\Abb}{\mathbb{A}}
\newcommand{\Fbb}{\mathbb{F}}
\newcommand{\Ibb}{\mathbb{I}}
\newcommand{\Nbb}{\mathbb{N}}
\newcommand{\Rbb}{\mathbb{R}}
\def\R{{\mathbb R}}
\newcommand{\Xbb}{\mathbb{X}}
\newcommand{\Zbb}{\mathbb{Z}}

\DeclareMathOperator*{\argmin}{arg\,min}

\newcommand{\MAP}{\operatorname{\textsc{map}}}
\newcommand{\range}{\operatorname{range}}
\renewcommand{\span}{\operatorname{span}}
\newcommand{\argmax}{\arg\!\max}
\begin{document}
\chapter{Surrogate Modelling for Gravitational Waveforms}
The highly anticipated detection of gravitational waves in 2015 from the coalescence of two stellar mass black holes heralded the beginning of a new era of astronomy. For the foreseeable future, compact binary coalescenses (CBCs) will remain the most detectable sources of gravitational waves. The the detection of these events requires high fidelity waveforms for use in the matched filtering pipeline. However, calculating these waveforms can take weeks to months of supercomputer resources, so adequately surveying the high dimensional parameter space is a computationally infeasible undertaking.

For a gravitational wave $h(t,\mu)$, where $\mu \in \bfR^8$ is a vector describing the binary source parameters, we seek a representation
\begin{equation}
h(t;\mu) \approx \sum_{i=1}^m A_i (\mu) e_i(t)
\end{equation}
for some basis $e_i$, and as small an $m$ as possible. 

Explore elementary effects.



\end{document}
