\documentclass[12pt]{report}

\usepackage{amsmath, amssymb, mathrsfs, graphicx, amsthm}
\usepackage[chapter]{algorithm}
\usepackage{algpseudocode}
\usepackage[T1]{fontenc}
\usepackage{hyperref}
%\usepackage{natbib}
%\algnotext{EndFor}
\addtolength{\textwidth}{100pt}
\addtolength{\oddsidemargin}{-50pt}
\addtolength{\evensidemargin}{-50pt}
%\linespread{1.2}

%\theorembodyfont{\rm}
\newtheorem{defn}{Definition}[section]

\newtheorem{thm}{Theorem}[section]
\newtheorem{lemma}[thm]{Lemma} %lemmas will be numbered using same counter as theorems


%\newenvironment{mydefn}{\begin{defn} \begin{upshape}}{\end{upshape} \end{defn}}
%\newenvironment{proof}
%       {\begin{flushleft} \begin{description}
%              \item \textit{\textbf{Proof:}}}
%        {\hfill\rule{2.1mm}{2.1mm}
%              \end{description}\end{flushleft}}

\def\pd{{\partial}}
\def\d{{\mbox{d}}}

\def\grad {{\nabla}}
\def\al{{\alpha}}
\def\be{{\beta}}
\def\ga{{\gamma}}
\def\eps{{\epsilon}}
\def\muv{{\overrightarrow{\mu}}}

\newcommand{\Abf}{\mathbf{A}}
\def\bfe{{\mathbf{e}}}
\newcommand{\Fbf}{\mathbf{F}}
\newcommand{\fbf}{\mathbf{f}}
\newcommand{\hbf}{\mathbf{h}}
\newcommand{\bfh}{\mathbf{h}}
\newcommand{\bfi}{\mathbf{h}}
\newcommand{\ibf}{\mathbf{h}}
\newcommand{\nbf}{\mathbf{n}}
\newcommand{\bfn}{\mathbf{n}}
\newcommand{\Pbf}{\mathbf{P}}
\def\bfp{{\mathbf{p}}}
\newcommand{\pbf}{\mathbf{p}}
\def\tbf{{\mathbf{t}}}
\newcommand{\Ubf}{\mathbf{U}}
\def\bfu{{\mathbf{u}}}
\newcommand{\Xbf}{\mathbf{X}}
\newcommand{\xbf}{\mathbf{x}}
\newcommand{\bfx}{\mathbf{x}}
\def\bfx{{\mathbf{x}}}
\newcommand{\Ybf}{\mathbf{Y}}
\newcommand{\ybf}{\mathbf{y}}
\def\bfy{{\mathbf{y}}}

\def\bfmu{{\mathbf{\mu}}}

\newcommand{\Ccal}{\mathcal{C}}
\newcommand{\Fcal}{\mathcal{F}}
\newcommand{\Hcal}{\mathcal{H}}
\newcommand{\Ical}{\mathcal{I}}
\def\Lie{{\mathcal{L}}}
\newcommand{\Pcal}{\mathcal{P}}
\newcommand{\Mcal}{\mathcal{M}}
\newcommand{\Dcal}{\mathcal{D}}
\newcommand{\Ncal}{\mathcal{N}}
\newcommand{\Ocal}{\mathcal{O}}
\newcommand{\Rcal}{\mathcal{R}}
\newcommand{\Ucal}{\mathcal{U}}
\newcommand{\Xcal}{\mathcal{X}}

\newcommand{\Abb}{\mathbb{A}}
\newcommand{\Cbb}{\mathbb{C}}
\newcommand{\Fbb}{\mathbb{F}}
\newcommand{\Ibb}{\mathbb{I}}
\newcommand{\Nbb}{\mathbb{N}}
\newcommand{\Rbb}{\mathbb{R}}
\def\R{{\mathbb R}}
\newcommand{\Vbb}{\mathbb{V}}
\newcommand{\Xbb}{\mathbb{X}}
\newcommand{\Zbb}{\mathbb{Z}}


\newcommand{\MAP}{\operatorname{\textsc{map}}}
\newcommand{\range}{\operatorname{range}}
\renewcommand{\span}{\operatorname{span}}
\newcommand{\argmax}{\arg\!\max}
\newcommand{\argmin}{\arg\!\min}
\begin{document}

\chapter{Reduced basis methods for parameterised PDEs}

%\chapter{Parameterised Models}
%We consider a model that can be described in four distinct steps. First, a parameter value is chosen from the $m$ dimensional parameter space. The initial condition is then found by solving a parameterised elliptic partial differential equation for the chosen parameter $p$. The time evolution is performed by solving a hyperbolic equation with the previously calculated initial condition, and the data is extracted from these solutions. 
%\begin{equation}
%\mathbf{p}\in \mathbb{R}^N \rightarrow u_0 \in H[0,1] \rightarrow u \in L_2 (\mathbb{R}, H) \rightarrow y \in L_2(\mathbb{R})
%\end{equation}
%We are interested in reconstructing the parameters $\bfp$ from (possibly noisy) output data $\bfy_d$. This involves finding the solution to an inverse problem. We pose this as a statistical inverse problem, and apply reduced basis methods for efficiency.

\section{Forward model}
Let $\bfp \in \mathcal{P} \subset \mathbb{R}^N$ be a parameter vector, and $\Omega \in \mathbb{R}^d$ be our function domain. The forward model $\mathcal{M}: \mathcal{P}\times \Omega \rightarrow $

\section{Statistical Inverse Problems}
In a deterministic setting, regularisation and optimisation techniques are used to find a single point estimate of the parameter $\bfp$. A statistical formaulation returns a probability density function over parameter space describing the relative likelihood of observation-consistent parameters, called the `posterior distribution', $\pi(\bfp| \bfy_d)$.

Here we infer properties of the probability distribution of model parameters $\bfp\in\R^N$ 
from observations $\bfy_d\in\R^m$. 
We adopt a Bayesian approach where a prior probability density $\pi(\bfp)$ of $\bfp$ is given. 
We also assume that our model of the observations provides us with
a likelihood function $\pi(\bfy_d\mid \bfp)$ of the data $\bfy_d$. 
The data $\bfy_d$ is treated as a random vector $\bfy$ with probability distribution $\pi(\bfy\mid \bfp)$ and we denote the expectation as
\begin{equation*}
\bfy(\bfp) := E(\bfy\mid \bfp).
\end{equation*}

Bayes' law leads to a formula for the posterior probability density of $\bfp$ as
\begin{equation}\label{eqn:bayes}
\pi(\bfp\mid \bfy_d) \propto \pi(\bfp)\,\pi(\bfy_d\mid \bfp).
\end{equation}
This probability density could then be further explored for example, in order to find marginals or
moments of the posterior or even the MAP (maximum a posteriori) estimate which is of the form
\begin{equation}\label{eqn:PMAPfull}
\bfp_{\MAP} = \argmax_{\bfp\in\R^N} \pi(\bfp\mid \bfy_d).
\end{equation}
In any case, the exploration of the posterior $\pi(\bfp\mid \bfy_d)$ requires its numerical evaluation typically many times. 
In our case each evaluation of the likelihood (and thus the posterior) 
requires an expensive numerical simulation.

\subsection{Reduced basis method}

In order to decrease the computational load required we reduce the parameter space from $\R^N$ to $\R^{N_r}$ as any exploration or computation in a lower dimensional space is cheaper. 
For this we generate a sequence of parameter vectors $\bfp_1, \bfp_2,\ldots$ in $\R^N$ which are defined using a greedy algorithm %as in \citep{Lieberman:2010:PSM:1958688.1958693}
\begin{equation}\label{eq:greedyOpt}
\bfp_{k+1} = \argmax_{\bfp\in\R^N} \left(\frac{1}{2}\|\bfy(\bfp)-\bfy(Q_k \bfp)\|^2 + \beta \log \pi(\bfp)\right),
\end{equation}
where the $Q_k:\R^N\rightarrow \R^N$ are orthogonal projections with 
$\range(Q_k)=\span\{\bfp_1,\ldots,\bfp_k\}$ for $k\geq 1$ and $Q_0:=0$. 
This choice introduces the new $\bfp_{k+1}$ to make sure that a large number of observations $\bfy_d$ can be approximated in the range of $Q_{k+1}$ while exploring parameter vectors which are sufficiently likely according to the prior.

\end{document}