%\documentclass[]{report}   % list options between brackets
%\usepackage{amsmath}
%\usepackage{amssymb}
%\usepackage{amsthm}
%\def\grad {{\nabla}}
%
%% type user-defined commands here
%
%\begin{document}

\chapter{State of the Art (this chapter is a total mess)}

\begin{itemize}
\item During inspiral and after the merger during the ringdown, can use perturbative methods (McWilliams- that's all he said about it) Due to the high computational cost of NR BBH simulations, the early inspiral can instead be modelled using approximate analytic techniques based on the post- Newtonian method, and the late inspiral and merger can be simulated with NR. PN results are accurate in the regime where the black holes are far apart and moving slowly. (Hinder?)
\item choices of $\alpha$ and $\beta$ lead to different numerical properties- for 4 decades most research was dedicated to finding stable choices (McWilliams). 
\end{itemize}

\section{Extreme Mass Ratios}
Usually in astrophysics, extreme mass ratios are on the order of $10^{6}$. These kinds of systems can be explored using perturbative methods, so in numerical relativity the focus is more around the $10^2$ level. The time scale of the orbit is $~M_1+M_2$, while the size of the time step is $M_small$, this makes it difficult \cite{Scheel2013}. 

The smallest mass ratio evolved to date is $q=10^{-2}$ by Lousto and Zlochower in 2011 \cite{LoustoZloch2011}. They used Cactus and the Einstein toolkit, with the Carpet mesh refinment driver to evolve the BBH system through the last two orbits, the merger and through to the Kerr black hole remnant. Their success was due to improvements in the moving puncture numerical technique.  

\section{Longest Waveforms}
The longest NR BBH waveform so far produced lasts for 15 orbits and includes the merger and ringdown phases, and is described in Boyle et al. [39] and Scheel et al. [40]. This waveform, from an equal mass binary of non-spinning black holes, was generated using the SpEC code and, due to its length and quoted accuracy, has been used in a number of studies comparing NR and PN results [41–46].As a result of this work, simulations of BBH mergers with mass ratios q = m1/m2 ≥ 1 (where m1 ≥ m2 are the masses of the individual holes) of q = 2 and dimensionless spins up to 0.4 are now possible with the SpEC code, and these are presented in Ref. [47]Additionally, simulations with dimensionless spins of 0.44 anti-aligned with the orbital angular momentum are presented in Chu et al. [48]. In a talk by H. Pfeiffer [49], a series of long unequal mass simulations performed with the SpEC code was presented, with 15 orbits up to q = 4 and 8 orbits up to q = 6. (Hinder?)

\section{Highest Spin}
Extreme spin $a=S/M^2 \leq 0.95$ by Lovelace, Scheel and Szilagyi 2011. Equal mass black holes with spin anti-aligned with orbital angular momentum (this reduces total angular momentum of the system). Something happens with Bowen York initial data that limits the spin to $\approx 0.93$ which is only 60\% of possible rotational energy (look into this). The authors fixed this by combining the conformal thin sandwich with a conformally curved metric. (McWilliams)
Mathematically, Kerr black holes have a maximum dimensionless spin of 1, and there is a good probability that highly spinning black holes exist in nature [57–59]. Accretion models suggest $~.95$, some observations suggest $>.98$. Typically, black holes with dimensionless spins as high as 0.6–0.8 can be evolved with only a moderate increase in computational cost over the non-spinning case. The theoretical maximum has almost been reached in Dain et al. [60], where ∼ 7.5 orbits of black holes with dimensionless spins 0.92 are evolved. 

\section{Christodoulou Memory}
Non-linear propagation of GW results in the emission of GWs generated by GWs, known as Christodoulou memory.  The effect is tiny, but potentially detectable by LISA. It's difficult to simulate because the errors in simulation need to be smaller than the already tiny effect. Pollney and Reisswig 2010 accurately calculated this effect using multipatch techniques and Cauchy characteristic extraction. (McWilliams)

\section{General Relativity Magneto Hydro Dynamics}
This work helps with finding EM counterparts to GW radiation. Studying Gamma Ray Bursts \cite{Schnetter2008}. Four codes capable of GRMHD: WhiskyMHD, SACRA, LSU-LIU-BYU-PI collaboration code, Illinois group code.

Samurai project: compare the waveforms from fice different codes for the last 4 orbits and merger of a binary, equal mass, non-spinning black holes in circular orbit. These codes were BAM, CCATIE, Hahndol, MayaKranc and SpEC. The first fours use the BSSN fomrulation of the Einstein equations while SpEC uses the Generalised Harmonic formulation.SpEC uses a pseudo-spectral evolution scheme, whereas the others use finite differencing method.

The XiRel project [63, 64] was started in or- der to improve the performance of the publicly available Carpet [36, 37] adaptive mesh refinement infrastructure. As a result of recent work, Carpet now scales efficiently up to 2048 processing cores, and as it is used for BBH simulations at AEI, GaTech, LSU, RIT and UIUC,



Hyperbolic BBH encounters can be thought of as orbits of eccentricity e > 1. In a Newtonian system, such a configuration would result in scattering of one black hole off the potential of the other, but in full GR, for a sufficiently small impact parameter, the black holes be- come gravitationally bound due to the emission of energy through gravitational waves and merge quickly [109]

Current evolutions using the SpEC code in the Generalised Harmonic formulation such as those in Refs. [39, 40] use an outer boundary condition designed to satisfy the Einstein constraint equations as well as to min- imise the incoming gravitational radiation [31, 149, 150], though these are not designed to be mathematically well- posed (a necessary condition for formal stability of the problem under small perturbations).

Mathematically, gravitational radiation from a BBH inspiral and merger is defined at future null infinity; i.e. the part of the spacetime towardswhich all null rays, such as light or GWs, propagate.GWs must be read off at a finite spatial radius, and this introduces an error in the waveform. An obvious solution is to compute the radiation at several radii and extrapolate the waveform to infinite radius as a function of retarded time fromthe source. It was shown in Hannam et al. Ref. [155] that this is not a trivial pro- cedure, and a detailed analysis and successful extrapo- lation of the inspiral portion was presented in Boyle et al. Ref. [39]. A failure to extrapolate waveforms from the SpEC code can lead to a phase error of 0.5 radians [39], or a mismatch of ∼ 10−2 [156], if using waveforms extracted at r = 50M only.

Instead of performing a traditional Cauchy evolution, where the solution is evolved along timelike directions, it is possible to evolve along null directions. This is called characteristic evolution. This cannot be done straightfor- wardly in the neighbourhood of the black holes as it leads to caustics in the solution, but it can be done in regions sufficiently far from them. The Cauchy-Characteristic Extraction (CCE) method combines the two approaches.

For numerical relativity evolution purposes, the four-dimensional BBH spacetime is usually foliated with three-dimensional spacelike slices. Each of these slices can be split into two regions with very different compu- tational requirements. The region around the black holes has a complicated geometry reflecting the shapes of the horizons, but the region far from the black holes con- sists only of gravitational radiation. Since the radiation propagates essentially radially, it requires a constant an- gular resolution to resolve it. The majority of BBH NR codes today use Cartesian-type coordinates everywhere in the grid. These have the advantage of simplicity, as only a single coordinate patch is required to cover the en- tire simulation domain. However, they are not efficient, as they lead to an increasing angular resolution with radius. 

While most NR codes use finite differencing methods with global Cartesian coordinates, the SpEC code uses multiple coordinate patches as well as spectral methods. For simple equations, these methods can be shown to be significantly more accurate (exponential rather than polynomial convergence) for a given computational cost, and indeed, the quoted accuracy and efficiency on the waveforms from the SpEC code is impressive.

\section{Cactus}
Flesh provides APIs for thorns to communicate with each other, performs administrative tasks and build and run time. Each thorn profides three configuration files (plus two optional files)


%\end{document}